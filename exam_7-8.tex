\documentclass[russian,]{article}
\usepackage[]{amsmath}
\usepackage{amssymb,amsmath}
\usepackage{ifxetex,ifluatex}
\usepackage{fixltx2e} % provides \textsubscript
\ifnum 0\ifxetex 1\fi\ifluatex 1\fi=0 % if pdftex
  \usepackage[T1, T2A]{fontenc}
  \usepackage[utf8]{inputenc}
\else % if luatex or xelatex
  \ifxetex
    \usepackage{mathspec}
  \else
    \usepackage{fontspec}
  \fi
  \defaultfontfeatures{Ligatures=TeX,Scale=MatchLowercase}
\fi
% use upquote if available, for straight quotes in verbatim environments
\IfFileExists{upquote.sty}{\usepackage{upquote}}{}
% use microtype if available
\IfFileExists{microtype.sty}{%
\usepackage[]{microtype}
\UseMicrotypeSet[protrusion]{basicmath} % disable protrusion for tt fonts
}{}
\PassOptionsToPackage{hyphens}{url} % url is loaded by hyperref
\usepackage[unicode=true]{hyperref}
\hypersetup{
            pdfborder={0 0 0},
            breaklinks=true}
\urlstyle{same}  % don't use monospace font for urls
\ifnum 0\ifxetex 1\fi\ifluatex 1\fi=0 % if pdftex
  \usepackage[shorthands=off,main=russian]{babel}
\else
  \usepackage{polyglossia}
  \setmainlanguage[]{}
\fi
\usepackage{longtable,booktabs}
% Fix footnotes in tables (requires footnote package)
\IfFileExists{footnote.sty}{\usepackage{footnote}\makesavenoteenv{long table}}{}
\usepackage{graphicx,grffile}
\makeatletter
\def\maxwidth{\ifdim\Gin@nat@width>\linewidth\linewidth\else\Gin@nat@width\fi}
\def\maxheight{\ifdim\Gin@nat@height>\textheight\textheight\else\Gin@nat@height\fi}
\makeatother
% Scale images if necessary, so that they will not overflow the page
% margins by default, and it is still possible to overwrite the defaults
% using explicit options in \includegraphics[width, height, ...]{}
\setkeys{Gin}{width=\maxwidth,height=\maxheight,keepaspectratio}
\IfFileExists{parskip.sty}{%
\usepackage{parskip}
}{% else
\setlength{\parindent}{0pt}
\setlength{\parskip}{6pt plus 2pt minus 1pt}
}
\setlength{\emergencystretch}{3em}  % prevent overfull lines
\providecommand{\tightlist}{%
  \setlength{\itemsep}{0pt}\setlength{\parskip}{0pt}}
\setcounter{secnumdepth}{0}
% Redefines (sub)paragraphs to behave more like sections
\ifx\paragraph\undefined\else
\let\oldparagraph\paragraph
\renewcommand{\paragraph}[1]{\oldparagraph{#1}\mbox{}}
\fi
\ifx\subparagraph\undefined\else
\let\oldsubparagraph\subparagraph
\renewcommand{\subparagraph}[1]{\oldsubparagraph{#1}\mbox{}}
\fi

% set default figure placement to htbp
\makeatletter
\def\fps@figure{htbp}
\makeatother

\usepackage[a4paper,left=20mm,right=10mm,top=20mm,bottom=20mm]{geometry}

\usepackage{amsgen, amsmath, amstext, amsbsy, amsopn, amsfonts, amsthm, thmtools,  amssymb, amscd, mathtext, mathtools}
\usepackage{versions}

\usepackage{float}
\restylefloat{table}

\usepackage{xfrac}

\usepackage{pifont}

\usepackage{xspace}

% Explain
\newcommand{\expl}[2]{\underset{\mathclap{\overset{\uparrow}{#2}}}{#1}}
\newcommand{\explup}[2]{\overset{\mathclap{\underset{\downarrow}{#2}}}{#1}}
\newcommand{\obrace}[2]{\overbrace{#1}^{#2}}
\newcommand{\ubrace}[2]{\underbrace{#1}_{#2}}

% Arrows
\newcommand{\Then}{\Rightarrow}
\newcommand{\Iff}{\Leftrightarrow}
\newcommand{\When}{\Leftarrow}
\newcommand{\Bydef}{\rightleftharpoons}
%\newcommand{\Divby}{\raisebox{-2pt}{\vdots}}
\DeclareRobustCommand{\Divby}{%
  \mathrel{\text{\vbox{\baselineskip.65ex\lineskiplimit0pt\hbox{.}\hbox{.}\hbox{.}}}}%
}

\DeclareMathOperator{\Char}{char}
\DeclareMathOperator{\Ker}{Ker}
\DeclareMathOperator{\Quot}{Quot}
\DeclareMathOperator{\Gal}{Gal}
\DeclareMathOperator{\Aut}{Aut}

% Mathbbs
\newcommand{\N}{\mathbb{N}}
\newcommand{\Z}{\mathbb{Z}}
\newcommand{\Q}{\mathbb{Q}}
\newcommand{\R}{\mathbb{R}}
\renewcommand{\C}{\mathbb{C}}

\renewcommand{\~}{\sim}
\renewcommand{\phi}{\varphi}
\newcommand{\ol}{\overline}
\newcommand{\cmark}{\ding{51}}
\newcommand{\xmark}{\ding{55}}
\newcommand{\y}{\cmark}
\newcommand{\x}{\xmark}

% Кавычки
\newcommand{\lgq}{\guillemotleft\nobreak\ignorespaces}
\newcommand{\rgq}{\guillemotright\xspace}

% Consider changing to sfrac
\newcommand{\bigslant}[2]{{\raisebox{.2em}{$#1$}\left/\raisebox{-.2em}{$#2$}\right.}}

\makeatletter
\newenvironment{sqcases}{%
  \matrix@check\sqcases\env@sqcases
}{%
  \endarray\right.%
}
\def\env@sqcases{%
  \let\@ifnextchar\new@ifnextchar
  \left\lbrack
  \def\arraystretch{1.2}%
  \array{@{}l@{\quad}l@{}}%
}
\makeatother

\makeatletter
\newenvironment{nocases}{%
  \matrix@check\sqcases\env@sqcases
}{%
  \endarray\right.%
}
\def\env@nocases{%
  \let\@ifnextchar\new@ifnextchar
  \def\arraystretch{1.2}%
  \array{@{}l@{\quad}l@{}}%
}
\makeatother

\newcommand{\nopandoc}[1]{#1} % hide LaTeX code from pandoc
\nopandoc{
	\let\Begin\begin
	\let\End\end
}


% Styles
\declaretheoremstyle[notefont=\bfseries,notebraces={}{},headpunct={},%
postheadspace={5px},headformat={\makebox[0pt][r]{\NAME\ \NUMBER\ }\setbox0\hbox{\ }\hspace{-\the\wd0}\NOTE}]{problemstyle}
\declaretheorem[style=problemstyle,numbered=no,name=№]{problem}

\declaretheoremstyle[notefont=\bfseries,notebraces={}{},headpunct={ },postheadspace={0px},qed=$\blacktriangleleft$,
headformat={\makebox[0pt][r]{\NAME\ }\setbox0\hbox{\ }\hspace{-\the\wd0}\NOTE},]{solutionstyle}
\declaretheorem[style=solutionstyle,numbered=no,name=$\blacktriangleright$]{solution}

\declaretheoremstyle[notefont=\bfseries,notebraces={}{},headpunct={.},postheadspace={4px}]{definitionstyle}
\declaretheorem[style=definitionstyle,numbered=yes,name=Опр.]{defn}



\date{}

\begin{document}

\begin{problem}[10 [Каргальцев]]
Теорема Гильберта о базисе

Нужно доказать, что если $K$ --- нетерово, то и $K[x]$ тоже нетерово (это и есть теорема Гильберта о базисе).
\end{problem}

\begin{solution}
Пусть есть цепочка строго вложеных в \(K[x]\) идеалов \(I_1 \subsetneq I_2 \subsetneq \ldots \subsetneq I_n \subsetneq \ldots\)

Положим \(I = \cup I_i\). Как неоднократно обсуждалось (\hyperlink{5.6}{5.6}, \hyperlink{8.2}{8.2}) \(I\) --- идеал.

Будем итеративно строить последовательность \(f_1, \ldots, f_n, \ldots \in K[x]\)

На \(i\)-м шаге будем выбирать \(f_i \in I \backslash (f_1, f_2, \ldots, f_{i - 1}): \deg f_i \to \min\).

(На первом шаге просто выберем \(f_i \in I: \deg f_1 \to \min\). Под \((f_1, \ldots, f_{i - 1})\) подразумевается идеал, порожденный соответствующими многочленами).

Корректность выбора (т.е что такое \(f_i\) существует) следует из того, что \(f_1, \ldots f_{i - 1} \in I_{i - 1} \Rightarrow (f_1, \ldots, f_{i - 1}) \subset I_{i - 1} \subsetneq I_i \subset I\).

Рассмотрим теперь старшие коэффициенты этих многочленов \(a_1, a_2, \ldots, a_n, \ldots\). Сразу заметим, что при \(i < j: I \backslash (f_1, \ldots, f_i) \supset I \backslash (f_1, \ldots, f_j) \Rightarrow \deg f_i \leqslant \deg f_j\).

Рассмотрим цепочку идеалов \((a_1) \subset (a_1, a_2) \subset \ldots \subset (a_1, \ldots, a_n) \subset \ldots\). Это последовательность вложенных идеалов из \(K\). Поскольку \(K\) --- нетерово, она стабилизируется, то есть существует такое \(N\), что \(a_{N + 1} \in (a_1, \ldots, a_N) \Rightarrow \; \exists \; b_1, b_2, \ldots b_N: a_{N + 1} = \sum\limits_{i = 1}^N b_i a_i\).

Рассмотрим \(f = f_{N + 1} - \sum\limits_{i = 1}^N b_i \cdot f_i \cdot x^{\deg f_{N + 1} - \deg f_i}\). (Все степени \(x\)-ов неотрицательны по замечанию выше). Степень \(f\) строго меньше степени \(f_{N + 1}\). С другой стороны, если \(f \in (f_1, \ldots, f_N) \Rightarrow f_{N + 1} \in (f_1, \ldots, f_N)\), что не так. Получили противоречие с минимальностью степени \(f_{N + 1}\).

То есть в \(K[x]\) не существует последовательности строго вложенных идеалов.

Пусть в \(K[x]\) есть последовательность вложенных идеалов, которая не стабилизируется. Тогда из нее можно выделить подпоследовательность строго вложенных идеалов. (Не стабилизируется равносильно тому, что \(\forall N \exists n > N: I_N \subsetneq I_n\)).

Получили, что \(K[x]\) нетерово, что и требовалось.
\end{solution}

\end{document}
