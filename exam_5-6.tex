\documentclass[russian,]{article}
\usepackage[]{amsmath}
\usepackage{amssymb,amsmath}
\usepackage{ifxetex,ifluatex}
\usepackage{fixltx2e} % provides \textsubscript
\ifnum 0\ifxetex 1\fi\ifluatex 1\fi=0 % if pdftex
  \usepackage[T1, T2A]{fontenc}
  \usepackage[utf8]{inputenc}
\else % if luatex or xelatex
  \ifxetex
    \usepackage{mathspec}
  \else
    \usepackage{fontspec}
  \fi
  \defaultfontfeatures{Ligatures=TeX,Scale=MatchLowercase}
\fi
% use upquote if available, for straight quotes in verbatim environments
\IfFileExists{upquote.sty}{\usepackage{upquote}}{}
% use microtype if available
\IfFileExists{microtype.sty}{%
\usepackage[]{microtype}
\UseMicrotypeSet[protrusion]{basicmath} % disable protrusion for tt fonts
}{}
\PassOptionsToPackage{hyphens}{url} % url is loaded by hyperref
\usepackage[unicode=true]{hyperref}
\hypersetup{
            pdfborder={0 0 0},
            breaklinks=true}
\urlstyle{same}  % don't use monospace font for urls
\ifnum 0\ifxetex 1\fi\ifluatex 1\fi=0 % if pdftex
  \usepackage[shorthands=off,main=russian]{babel}
\else
  \usepackage{polyglossia}
  \setmainlanguage[]{}
\fi
\usepackage{longtable,booktabs}
% Fix footnotes in tables (requires footnote package)
\IfFileExists{footnote.sty}{\usepackage{footnote}\makesavenoteenv{long table}}{}
\usepackage{graphicx,grffile}
\makeatletter
\def\maxwidth{\ifdim\Gin@nat@width>\linewidth\linewidth\else\Gin@nat@width\fi}
\def\maxheight{\ifdim\Gin@nat@height>\textheight\textheight\else\Gin@nat@height\fi}
\makeatother
% Scale images if necessary, so that they will not overflow the page
% margins by default, and it is still possible to overwrite the defaults
% using explicit options in \includegraphics[width, height, ...]{}
\setkeys{Gin}{width=\maxwidth,height=\maxheight,keepaspectratio}
\IfFileExists{parskip.sty}{%
\usepackage{parskip}
}{% else
\setlength{\parindent}{0pt}
\setlength{\parskip}{6pt plus 2pt minus 1pt}
}
\setlength{\emergencystretch}{3em}  % prevent overfull lines
\providecommand{\tightlist}{%
  \setlength{\itemsep}{0pt}\setlength{\parskip}{0pt}}
\setcounter{secnumdepth}{0}
% Redefines (sub)paragraphs to behave more like sections
\ifx\paragraph\undefined\else
\let\oldparagraph\paragraph
\renewcommand{\paragraph}[1]{\oldparagraph{#1}\mbox{}}
\fi
\ifx\subparagraph\undefined\else
\let\oldsubparagraph\subparagraph
\renewcommand{\subparagraph}[1]{\oldsubparagraph{#1}\mbox{}}
\fi

% set default figure placement to htbp
\makeatletter
\def\fps@figure{htbp}
\makeatother

\usepackage[a4paper,left=20mm,right=10mm,top=20mm,bottom=20mm]{geometry}

\usepackage{amsgen, amsmath, amstext, amsbsy, amsopn, amsfonts, amsthm, thmtools,  amssymb, amscd, mathtext, mathtools}
\usepackage{versions}

\usepackage{float}
\restylefloat{table}

\usepackage{xfrac}

\usepackage{pifont}

\usepackage{xspace}

% Explain
\newcommand{\expl}[2]{\underset{\mathclap{\overset{\uparrow}{#2}}}{#1}}
\newcommand{\explup}[2]{\overset{\mathclap{\underset{\downarrow}{#2}}}{#1}}
\newcommand{\obrace}[2]{\overbrace{#1}^{#2}}
\newcommand{\ubrace}[2]{\underbrace{#1}_{#2}}

% Arrows
\newcommand{\Then}{\Rightarrow}
\newcommand{\Iff}{\Leftrightarrow}
\newcommand{\When}{\Leftarrow}
\newcommand{\Bydef}{\rightleftharpoons}
%\newcommand{\Divby}{\raisebox{-2pt}{\vdots}}
\DeclareRobustCommand{\Divby}{%
  \mathrel{\text{\vbox{\baselineskip.65ex\lineskiplimit0pt\hbox{.}\hbox{.}\hbox{.}}}}%
}

\DeclareMathOperator{\Char}{char}
\DeclareMathOperator{\Ker}{Ker}
\DeclareMathOperator{\Quot}{Quot}
\DeclareMathOperator{\Gal}{Gal}
\DeclareMathOperator{\Aut}{Aut}

% Mathbbs
\newcommand{\N}{\mathbb{N}}
\newcommand{\Z}{\mathbb{Z}}
\newcommand{\Q}{\mathbb{Q}}
\newcommand{\R}{\mathbb{R}}
\renewcommand{\C}{\mathbb{C}}

\renewcommand{\~}{\sim}
\renewcommand{\phi}{\varphi}
\newcommand{\ol}{\overline}
\newcommand{\cmark}{\ding{51}}
\newcommand{\xmark}{\ding{55}}
\newcommand{\y}{\cmark}
\newcommand{\x}{\xmark}

% Кавычки
\newcommand{\lgq}{\guillemotleft\nobreak\ignorespaces}
\newcommand{\rgq}{\guillemotright\xspace}

% Consider changing to sfrac
\newcommand{\bigslant}[2]{{\raisebox{.2em}{$#1$}\left/\raisebox{-.2em}{$#2$}\right.}}

\makeatletter
\newenvironment{sqcases}{%
  \matrix@check\sqcases\env@sqcases
}{%
  \endarray\right.%
}
\def\env@sqcases{%
  \let\@ifnextchar\new@ifnextchar
  \left\lbrack
  \def\arraystretch{1.2}%
  \array{@{}l@{\quad}l@{}}%
}
\makeatother

\makeatletter
\newenvironment{nocases}{%
  \matrix@check\sqcases\env@sqcases
}{%
  \endarray\right.%
}
\def\env@nocases{%
  \let\@ifnextchar\new@ifnextchar
  \def\arraystretch{1.2}%
  \array{@{}l@{\quad}l@{}}%
}
\makeatother

\newcommand{\nopandoc}[1]{#1} % hide LaTeX code from pandoc
\nopandoc{
	\let\Begin\begin
	\let\End\end
}


% Styles
\declaretheoremstyle[notefont=\bfseries,notebraces={}{},headpunct={},%
postheadspace={5px},headformat={\makebox[0pt][r]{\NAME\ \NUMBER\ }\setbox0\hbox{\ }\hspace{-\the\wd0}\NOTE}]{problemstyle}
\declaretheorem[style=problemstyle,numbered=no,name=№]{problem}

\declaretheoremstyle[notefont=\bfseries,notebraces={}{},headpunct={ },postheadspace={0px},qed=$\blacktriangleleft$,
headformat={\makebox[0pt][r]{\NAME\ }\setbox0\hbox{\ }\hspace{-\the\wd0}\NOTE},]{solutionstyle}
\declaretheorem[style=solutionstyle,numbered=no,name=$\blacktriangleright$]{solution}

\declaretheoremstyle[notefont=\bfseries,notebraces={}{},headpunct={.},postheadspace={4px}]{definitionstyle}
\declaretheorem[style=definitionstyle,numbered=yes,name=Опр.]{defn}



\date{}

\begin{document}

\begin{problem}[1 (1.4) [Каргальцев]]
Для любого числа $u \in \C$ определим множество $\Z[u] = \cup_{n = 0}^{\infty} \{a_0 + a_1u + \ldots + a_nu^n | a_0, a_1, \ldots, a_n \in \Z\}$.
\end{problem}

\begin{solution}
а) Докажите, что \(\Z[u]\) является областью целостности.

То, что \(\Z[u]\) кольцо проверяется непосредственно. Поскольку \(\Z[u] \subset \C\) и \(\C\) --- область целостности (\it{потому что $\C$ --- поле}), то и \(\Z[u]\) область целостности.

б) При каких \(u \in \C\) данное \(\Z[u]\) ``конечномерно над \(\Z\)'', то есть найдётся такое \(N\), что \(\Z[u] = \cup_{n = 0}^{\infty} \{a_0 + a_1u + \ldots + a_nu^N | a_0, a_1, \ldots, a_N \in \Z\}\)?

Покажем, что \(\Z[u]\) ``конечномерно над \(\Z\)'', \(\Iff \exists f \in \Z[x]: f(u) = 0\), \(f \ne 0\) и старший коэффициент \(f(x)\) равен \(1\) \(\,\,\,(*)\).

\(\Then\)

Поскольку \(u^{N + 1} \in \Z[u] \Then \exists a_0, \ldots, a_N \in \Z: u^{N + 1} = \sum\limits_{0}^{N} a_ku^k \Then u\) --- корень \(f(x) = x^{N + 1} - \sum\limits_{0}^{N} a_kx^k\)

\(\When\)

Пусть \(u\) --- корень многочлена \(f(x) = u^{N} + \sum\limits_{0}^N a_kx^k\), удовл. условию \((*)\). Тогда \(u^N\) выражается через меньшие степени. (\(u^N = -\sum\limits_{0}^{N - 1}a_ku^k\))

Индукцией по \(k \geqslant N\) легко показать, что \(u^k\) выражается через \(1, u, \ldots u^{N - 1}\).

(\(u^{k + 1} = u \cdot u^{k} \stackrel{\textup{предположение индукции}}{=} u \cdot (\sum\limits_0^{N - 1} b_ku^k) = (\sum\limits_1^{N - 1} b_{k - 1} u^k) + b_{N - 1}u^N \stackrel{\textup{база индукции}}{=} (\sum\limits_1^{N - 1} b_{k - 1} u^k) + b_{N - 1}\sum\limits_{0}^{N - 1}-a_ku^k\)

\end{solution}

\begin{problem}[2. (1.2 )]
Для комплексного числа $z \in C$ введём норму $N(z) = |z|^2$.

а) $N(zw) = N(z)N(w)$.

Для каждого $z \in D$:

б) Верно ли, что $N(z)$ — натуральное число?

в) Верно ли, что $N(z) = 1 \Leftrightarrow z$ — обратим?
\end{problem}

\begin{solution}

а) Просто проверим: \(N(zw)= N(a_z + b_zi)(a_w + b_wi)) = N(a_za_w - b_zb_w + (a_zb_w + a_wb_z)i)=\)
\((a_za_w-b_zb_w)^2 + (a_zb_w + b_za_w)^2 =\)
раскрыли скобки
\(= (a_z^2 + b_z^2)(a_w^2 + b_w^2) = N(z)N(w)\)

б) Заметим, что \(\Z[i] = \{a + bi\,|\,a,b \in \Z\}\)

Значит, \(|a + bi| = a^2 + b^2 \in |N\). Аналогично:

\(\Z[2i] = \{a + 2bi\,|\,a,b \in \Z\} \Rightarrow |a + 2bi| = a^2 + 4b^2 \in \N\)

\(\Z[\sqrt{2}i] = \{a + \sqrt{2}bi\,|\,a,b \in \Z\} \Rightarrow |a + \sqrt{2}bi| = a^2 + 2b^2 \in \N\)

\(\Z[\sqrt{3}i] = \{a + \sqrt{3}bi\,|\,a,b \in \Z\} \Rightarrow |a + \sqrt{3}bi| = a^2 + 3b^2 \in \N\)

в) \(\Then\)

\(N(z) = a^2 + b^2 = 1\)

\(\frac{1}{z} = \frac{1}{a+bi} = \frac{a-bi}{a^2 + b^2} = \frac{a-bi}{1} = a-bi = \overline{z}\),
а \(z\) и \(\overline{z}\) одновременно лежат в \(D\), значит \(\exists z^{-1} = \overline{z}\).

\(\When\)

\(zz^{-1} = 1 \Rightarrow \begin{cases} N(zz^{-1}) = N(z)N(z^{-1}) = 1 \\ N(z) = a^2 + b^2 \geqslant 1  \end{cases} \Rightarrow N(z) = 1\)

\end{solution}

\begin{problem}[3]
Пример нефакториального кольца вида $Z[u]$.
\end{problem}

\begin{solution}
Пример: \(\Z[2i]\) не является факториальным кольцом, потому что \(4 = 2\cdot2 = (2i)(-2i)\),
но при этом \(2 \nsim 2i\) -- противоречие с единственностью разложения в факториальном кольце.

Еще пример: \(\Z[\sqrt{3}i]\) (аналогичное рассуждение \(4 = 2\cdot 2 = (1+\sqrt{3}i)(1-\sqrt{3}i)\)).
\end{solution}

\begin{problem}[4 (2.7) [Каргальцев]]
Простой элемент области целостности является неразложимым.
\end{problem}

\begin{solution}
Пусть \(p\) --- простой и \(p = xy \Then x | p \land y | p\) . Из определения
простоты \(p | x \lor p | y\). Но тогда или \(x | p \land p | x\), или
\(y | p \land p | y\). Тогда \(p \sim y \lor p \sim x \Then y \in K^* \lor x \in K^*\),
то есть \(p\) --- неразложимый.
\end{solution}

\begin{problem}[5 (2.8)]
В факториальном кольце любой неразложимый элемент является простым.
\end{problem}

\begin{solution}
Пусть \(x=ab\) -- неразложимый. \(x = ab \Then x\,|\,ab\).

\(x\) неразложимый, значит б.о.о. \(a\in K^*\). Тогда в силу единственности разложения
\(x = ab = ap_1\ldots p_k \Then x \sim b \Then x\,|\,b\).
\end{solution}

\begin{problem}[6 (часть 2.9) [Каргальцев]]
$K$ --- евклидово кольцо. Верно ли, что если для $a, b \ne 0$ выполнено равенство $N(ab) = N(a)$, то $b$ обратим?
\end{problem}

\begin{solution}

Поделим \(a\) с остатком на \(ab\):

\[a = abq + r: r=0 \lor N(r) < N(ab)\].
\[r = a(1 - bq)\].

Если \(r=0\), то \(bq = 1\) и \(b\) обратим. Иначе \(N(ab) > N(r) = N(a(1 - bq)) \geqslant N(a) = N(ab)\). Противоречие.
\end{solution}

\begin{problem}[7 (2.10)]
Геометрический способ доказательства того, что $\Z[i]$, $\Z[\omega]$ — евклидово кольцо.
\end{problem}

\begin{solution}
ВСТАВИТЬ КАРТИНКУ
Пусть \(a, b \in \Z[i]\). Поделим \(a\) на \(b\) с остатком:

\(a = pb + q\).

Надо доказать, что если \(q \neq 0\), то \(N(q) < N(b)\). Рассмотрим точку \(\frac{a}{b}\),
пусть ближайший к ней узел в решетке \(p\), тогда \(\frac{a}{b} = p + \frac{q}{b}\).
Но \(\frac{q}{b}\) по модулю меньше половины диагонали единичного квадрата
\(\left|\frac{q}{b}\right| \leqslant \left|\frac{\sqrt{2}}{2}\right| \leqslant 1\),
т.е. \(|q|^2 < |b|^2 \Then N(q) < N(b)\), если \(\frac{q}{b}\) не совпадает с центром квадрата.

(TODO иначе)

\(\Z[\omega]\) аналогично.
\end{solution}

% -----------------------------------------------------------

\begin{problem}[9 (3.1)]
а)~Если $p$ -- простое целое число и существует такое $z \in D$, что 
$N(z) = p$, то $z$ — неразложимый элемент.

б)~Если $p$ -- простое целое число и не существует такого $z \in D$, 
что $N(z) = p$, то $p$ -- неразложимый элемент.

в)~Если $D$ -- факториальное кольцо, то для любого неразложимого элемента
$z \in D$ либо $N(z) = p$, либо $z \sim p$ для некоторого целого простого числа $p$.

\end{problem}

\begin{solution}
а)~Имеем: $z \in D$, $N(z) = p$. Пусть $z=bc$, тогда $N(z) = N(b)N(c) = p \Then$ $N(b) = 1$ или $N(c) = 1$, т.е $b \in D^*$ или $с \in D^* \Then$ $z$ -- неразложим (исп. задачу 2в)

б)~Пусть $p = bc$. Тогда $N(p) = N(b)N(c) = p^2$. Два случая:

\begin{itemize}
	\item $N(b) = N(c) = p$ --- невозможно по условию
	\item $N(a) = 1,\,\,N(b) = p^2$ или  $N(a) = p^2,\,\,N(b) = 1 \Then$ $b \in D^*$ или $c \in D^*$ (исп. задачу 2в).
\end{itemize}

в)~$N(z) = z\overline{z} = p_1^{k_1} \cdot \ldots \cdot p_m^{k_m}$ (в силу факториальности кольца). $z$ неразложим $\Then \exists i\colon p_i \vdots z \Then zk=p_i$

$N(p_i) = p_i^2=N(z)N(k)\Then$ либо $N(z)=p_i\Then$ $z$ -- неразложим, либо $N(z) = p_i^2\Then z \sim p_i$.
 
\end{solution}

% --------------------------------------------------------------

\begin{problem}[10 (2.7) [Каргальцев]]
	Если $z \in D$, $z | x$, и $N(z) = N(x)$, то $z \sim x$.
\end{problem}

\begin{solution}
	Пусть \(x = yz\). Тогда \(N(yz) = N(z) \Then y\) обратим (по №6) и, значит, \(x \sim z\).
\end{solution}

% ------------------------------------------------------------------

\begin{problem}[11 (3.3) [Каргальцев]]
	(Простые гауссовы числа) Пусть $p$ --- простое целое число.
\end{problem}

\begin{solution}
	
	а) Если \(p\) = \(4k + 3\), то \(p\) --- неразложим в \(\Z[i]\).
	
	Если \(p\) разложим, тогда \(p = z\overline{z} = Re^2z + Im^2z\). Но число, дающее остаток 3 при делении на 4 не быть представлено в виде суммы двух квадратов (квадраты дают остаток 1 при делении на 4).
	
	б) Если \(p = 4k + 1\), то \(p\) --- разложим в \(\Z[i]\).
	
	Если \(p = 4k + 1\), то \(-1\) --- вычет по модулю \(p\), т. е \(\exists x \in \Z: p| x^2 + 1 \Then p | (x + i)(x - i)\). Если \(p\) --- неразложим, тогда \(p\) --- прост и либо \(p| (x + i)\), либо \(p | (x - i)\). 
	
	\begin{itemize}
		\item $p| (x + i)\Then x + i = p(c + di)\Then 1 = pd \Then p\,|\,1$ -- плохо.
		\item $p| (x - i)\Then x - i = p(c + di)\Then -1 = pd \Then p\,|\,1$ -- плохо.
	\end{itemize}
	Значит, \(p\) разложим.
	
	в) Если \(p = 4k + 1\), то \(p = z\overline{z}\), где \(z\) --- неразложим в \(\Z[i]\).
	
	Следует из предыдущего пункта и пункта г) предыдущей задачи.
	
	г) Неразложимые элементы \(\Z[i]\), не описанные в предыдущих пунктах --- \(\pm 1 \pm i\).
	
	Неразложимые элементы, не описанные в предыдущих задачах могут иметь норму или 2, или 4. Норму 4 имеет только \(2\) и ассоциированные с ней, но \(2 = (1 + i)(1 - i)\).
	
	С другой стороны, \(N(\pm 1 \pm i) = 2\), то есть силу пункта в) предыдущей задачи \(\pm 1 \pm i\) неразложимы.
\end{solution}

% ------------------------------------------------------------

\begin{problem}[12 (3.10)]
Евклидово кольцо является кольцом главных идеалов.
\end{problem}

\begin{solution}
Пусть $K$ -- евклидово кольцо, $a \in K$, причем \[N(a) = \min_{x\in K \setminus \{0\}}N(x)\].

Предположим, что $K \neq (a) \Then \exists b\in K\setminus (a) \Then b = aq + r$, где либо $r = 0$, либо $N(r) < N(a)$.

\begin{itemize}
	\item $r=0\Then b=aq\Then b \in (a)$ -- противоречие
	\item $N(r) < N(a) \Then r = b-aq \in I$ -- противоречие с минимальностью нормы $a$.
\end{itemize}
\end{solution}

% ------------------------------------------------------------

\begin{problem}[13 (3.13)]
	
Пусть $D = \Z[i]$ или $\Z[\omega]$. 

а) Верно ли, что из $a | b$ следует, что $N (a) | N (b)$? 

б) Верно ли, что из $\textup{НОД}(N (a), N (b)) = 1$, следует $\textup{НОД}(a, b) = 1$? 

в) Пусть $\textup{НОД}(N (a), N (b)) = p$ -- простое целое число, причём
$p \not| a$, $p \not| b$. Тогда $p$ -- разложим, и если $p = z\overline{z}$, то либо $z$ и $\overline{z}$ порождает идеал $(a, b)$, либо $z$ делит одно
из этих чисел, а $\overline{z}$ — другое.
	
\end{problem}
\begin{solution}
а)~Верно. $a\,|\,b \Then b=ak \Then N(b) = N(a)N(k)$ (по свойству нормы в $D$) $\Then N(a)\,|\,N(b)$

б)~$\textup{НОД}(N (a), N (b)) = p$

Допустим, что $\textup{НОД}(a, b) = k \notin D^*$. 

Тогда $a = kx, b = ky$, и
\begin{equation}
\left. \begin{gathered}
N(a) = N(kx) = N(k)N(x) \\
N(b) = N(ky) = N(k)N(y)
\end{gathered} \right \} \Then \textup{НОД}(N (a), N (b)) \ne 1
\end{equation} -- противоречие. Значит, $\textup{НОД}(a, b) = 1$.

в)~$\textup{НОД}(N (a), N(b)) = p$

Покажем, что $p$ -- разложим. $N(a) = a\overline{a} = pt, p \not| \,a$, $p$ -- простое число $\Then$  допустим, что $p$ неразложима: $p\,|\, \overline{a}$ (по свойству факториального кольца). Тогда $\overline{a} = x - iy\,\vdots\,p \Leftrightarrow x \,\vdots\,p, y \,\vdots\,p \Then a \,\vdots\,p$ -- противоречие $\Then p$ -- разложим.

\begin{equation}
\left\{
\begin{gathered} 
a\overline{a}  = z\overline{z}k \\
b\overline{b} = z \overline{z}l
\end{gathered}
\right. \Then
\left[
\begin{gathered} 
a\,\vdots\,z \textup{ и } b\,\vdots\,\overline{z} \\
a\,\vdots\,\overline{z} \textup{ и } b\,\vdots\,z \\
a\,\vdots\,z \textup{ и } b\,\vdots\,z
\end{gathered}
\right.
\end{equation} 

В последнем случае идеал $(t) = (a,b) \subseteq (z, \overline{z})$ -- очевидно.
Докажем в обратную сторону, что $(z, \overline{z})\subseteq (a,b)$

$z\overline{z} = a\overline{a}\xi + b\overline{b}\eta \,\vdots\,t$

\end{solution}


% ------------------------------------------------------------

\begin{problem}[14 (3.14)]
Умение находить порождающий элемент идеала в кольце $\Z[i]$.
\end{problem}
\begin{solution}
Возможный вариант решения: найдем нормы двух чисел, потом найдем $n$ -- НОД этих норм. После этого переберем все числа, которые имеют норму $n$ и проверим их на то, что они являются порождающим элементом. При этом искать можно только в первой четверти комплексной плоскости (т.к. найдя одно число, получаем сразу 4 поворотами на $\pi / 2$). Если не один из них не подойдет, то проделаем то же самое со всеми делителями $n$ в порядке уменьшения модуля, пока не дойдем до $1$.

Рассмотрим пример:

3.14. Найти порождающий элемент $(11+7i, 18-i)$ в $\Z[i]$.
\textit{Решение.} Заметим, что $\Z[i]$ -- евклидово кольцо, значит, оно является КГИ $\Then$ все идеалы главные $\Then$ идеал $(11+7i, 18-i)$ порождается одним элементом $(t)$. Найдем этот элемент.

$N(11+7i) = 170$

$N(18-i) = 325$

$\textup{НОД}(170, 325) = 5$.
	
Перебором выясняем, что в первой четверти числу с  нормой $5$ соответствуют два числа: $1 + 2i$ и $2 + i$.

Заметим, что $1 + 2i$ не может быть порождающим элементом:

$\frac{18-i}{1+2i} = \frac{(18-i)(1-2i)}{(1+2i)(1-2i)} = \ldots = \frac{16}{5} - \frac{37}{5}i \notin \Z[i]$

С $2+i$ тоже плохо:

$\frac{11+7i}{2+i} = \frac{29}{5} + \frac{3}{5}i \notin \Z[i]$.

Следовательно, среди чисел с нормой $5$ нет $\textup{НОД}(a, b) \Then $ его норма $1 \Then (11+7i, 18-i) = (1)$.   
\end{solution}


% ------------------------------------------------------------

\begin{problem}[?? [Каргальцев]]
\end{problem}

\begin{solution}
а) Если \(z\) --- неразложимый элемент \(D\), то существует такое простое целое число \(p\), что \(N(z) = p\) или \(N(z) = p^2\)

\(N(z) = z\overline{z}\). Разложим \(N(z)\) в произведение простых как натуральное число:

\(z\overline{z} = N(z) = p_1^{\alpha_1} \cdot \ldots \cdot p_n^{\alpha_n}\).

Так как \(z\) неразложим, а \(D\) --- евклидово, то \(z\) --- прост, значит \(\exists k: z | p_k\).

\(p_k = zu \Then p_k = \overline{p_k} = \overline{z}\overline{u} \Then \overline{z} | p_k \Then N(z) | p_k^2 \Then N(z) = 1, p_k\) или \(p_k^2\). Но так как если \(N(z) = 1\), то \(z\) --- обратим (а, следовательно, неразложим), то \((z) = p_k \lor N(z) = p_k^2\).

б) Если \(z\) --- неразложимый элемент \(D\) и \(N(z) = p^2\), то \(z \sim p\).

Пусть \(\overline{z} = ab \Then z = \overline{a} \overline{b} \Then \overline{z}\) --- неразложим.

\(z \overline{z} = N(z) = p \cdot p\). В силу единственности разложения на неразложимые, \(z \sim p\).

в) Если \(N(z) = p\), то \(z\) --- неразложимый элемент \(D\).

в \(D a|b \Then N(a) | N(b)\).

Пусть \(a | z \Then N(a) | N(z)\). В силу простоты \(N(z)\) либо \(N(a) = 1\) и, следовательно, \(a\) --- обратимый, либо \(N(a) = N(z)\) и тогда \(a \sim z\). То есть \(z\) неразложим.

г) Пусть \(p\) --- простое целое число. Тогда есть два варианта: либо \(p\) неразложимо в \(D\), либо \(p\) = \(z\overline{z}\), где \(z\) -- неразложимо в \(D\). Таким образом описываются все неразложимые элементы \(D\).

Пусть \(p\) разложимо в \(D\). Тогда найдется такой неразложимый \(z: z|p\). Поскольку \(z\) не ассоциирован с \(p\), \(N(z) \ne N(p) \Then N(z) = p\). Тогда \(z\) -- неразложимый и \(z\overline{z} = N(z) = p\).

Любой неразложимый элемент \(D\) --- либо простое целое число, либо его норма --- простое целое число.
\end{solution}

% ------------------------------------------------------------

\begin{problem}[25 [Каргальцев]]
Докажите, что в кольце главных идеалов любая возрастающая цепочка идеалов

$$ (a_1) \subset (a_2) \subset \ldots \subset (a_n) \subset \ldots $$

стабилизируется, то есть найдется такое $k$, то $(a_k) = (a_{k + 1}) = \ldots$
\end{problem}

\begin{solution}
Поскольку \((a_i) \subset (a_{i + 1}) \Then a_{i + 1} | a_i\).

Возьмем \(I = \cup_{k = 1}^{\infty} (a_k)\). покажем, что \(I\) -- идеал. Пусть \(a \in I, b \in I \Then \exists k_1, k_2: a \in (a_{k_1}), b \in (a_{k_2})\). Тогда положим \(k = max(k_1, k_2)\). \(a, b \in (a_k) \Then (a + b) \in (a_k) ((a_k) \textup{--- идеал}) \Then (a + b) \in I\). Анологично \(\forall x \in K xa \in (a_k) \Then xa \in I\).

Поскольку \(K\) --- КГИ, то существует \(x: I = (x)\). \(x \in I \Then \exists k: x \in (a_k)\). Но \(a_k \in (x)\). Тогда \(x | a_k \land a_k | x \Then x \sim a_k\). Но в силу вложенности это верно и для всех \(j > k\), то есть \(\forall j \geqslant k a_j \sim a_k \Then (a_j) = (a_k)\). То есть цепочка действительно стабилизируется.
\end{solution}

\end{document}
